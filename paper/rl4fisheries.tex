\documentclass[floatfix,nofootinbib,longbibliography,notitlepage]{revtex4-1}


%basic packages
\usepackage{xcolor}
\usepackage[T1]{fontenc}
\usepackage{hyperref}

\usepackage{amsthm}
\usepackage{amsmath}
\usepackage[mathscr]{euscript}
\usepackage{bbm}
\usepackage{times}
\usepackage{amsfonts}
\usepackage{amssymb}
\usepackage{thmtools}
%\usepackage{thm-restate}
%\usepackage{enumitem}
%\usepackage{amscd}
%\usepackage{graphicx}
%\usepackage{braket}
\usepackage{physics}
%\usepackage{multirow}
%\usepackage{tikz}
%\usetikzlibrary{quantikz}

%colors
\definecolor{Fabulous}{RGB}{220,0,100}

%comments
\newcommand{\fm}[1]{{\color{Fabulous}\bf [FMM: #1]}}

\newcommand\numberthis{\addtocounter{equation}{1}\tag{\theequation}}

\def\textfmsy{$F_{\text{MSY}}$}
\def\fmsy{F_{\text{MSY}}}
\def\bmsy{B_{\text{MSY}}}

\def\bsurv{B_{\text{survey}}}
\def\vsurv{v_{\text{survey}}}
\def\nsurv{N_{\text{survey}}}
%
\def\vharv{v_{\text{harvest}}}
%
\def\ssb{\text{SSB}}

\def\obs{\text{obs}}

\def\eg{e.g.}


%handy macros
% eq + tab + enter: go to equation env
% align + tab + enter: similar but align env

\begin{document}

\title{When do size/age composition observations improve harvest control decisions? 
A machine learning case study on spasmodic age-structured populations}

\author{Felipe Montealegre-Mora}
\email{felimomo@berkeley.edu}
\affiliation{\
Department of Environmental Science, Policy and Management, University of California at Berkeley,  Berkeley, California, 94720, USA}
\affiliation{\ 
Eric and Wendy Schmidt Center for Data Science and Environment, University of California at  Berkeley, Berkeley, California, 94720, USA}

\author{Carl Boettiger}
\affiliation{\
Department of Environmental Science, Policy and Management, University of California at Berkeley,  Berkeley, California, 94720, USA}

\author{Carl J. Walters}
\affiliation{\
Institute for the Oceans and Fisheries, University of British Columbia, Vancouver, British Columbia, V6T 1Z4, Canada
}

\author{Christopher L. Cahill}
\affiliation{\
Quantitative Fisheries Center, Department of Fisheries and Wildlife, Michigan State University, East Lansing, Michigan, 48824, USA
}

\thanks{...
}

\date{\today}

\begin{abstract}
	In fishery science, harvest management of size-structured stochastic populations is a long-standing and  difficult problem. Rectilinear precautionary policies based on biomass and harvesting reference points have now become a standard approach to this problem. However, these policies are ultimately based on the theory of maximum sustainable yield and are theoretically grounded on the assumption that surplus production is stable in the fishery. This paradigm is challenged in fisheries with highly variable, spasmodic, recruitment. In this paper we explore the problem of designing harvest control rules for partially observed, age-structured, spasmodic fish populations using tools from reinforcement learning (RL) and Bayesian optimization. Our focus is on the case of Walleye fisheries in Alberta, Canada, whose highly variable recruitment dynamics have perplexed managers and ecologists. We optimized and evaluated policies using several complementary performance metrics. The main questions we addressed were: 1. How do standard policies based on reference points perform relative to numerically optimized policies? 2. Can an observation of mean fish weight, in addition to stock biomass, aid policy decisions? 
\end{abstract}

\maketitle

\tableofcontents
% so that the table of contents doesnt look weird:
\makeatletter
\let\toc@pre\relax
\let\toc@post\relax
\makeatother 

%
%
%
\section{Introduction}

Over the past two decades a wealth of literature has emerged that is aimed at solving sequential decision problems in diverse fields such as engineering, robotics, and control theory.  
This work, collectively referred to as Reinforcement Learning (RL), has now advanced to the level needed to outperform human experts in many fields \cite{sutton-rl,bertsekas-rl}.  
For example, the application of RL has revolutionized games like chess, where the world’s top chess engines now almost always defeat the greatest chess players in the world. 
Beyond simply outperforming experts, RL offers a fresh perspective on previously unsolvable problems—and in the case of chess, top players now incorporate strategies discovered by engines into their own repertoires that were once considered nonsensical by humans. 
Lessons like these seem relevant to fisheries scientists, as a number of sustainability problems lie at the intersection of age-structured population dynamics and sequential decision making under uncertainty \cite{walters-hilborn-1978}, and for which methods like dynamic programming or analytical approaches break down due to computational costs or scalability issues related to the “curse of dimensionality.”  
It is in these complex decision-making contexts that RL shows much promise, and might serve as a useful guide for improving our collective intuition surrounding the design of feedback policies.

In the absence of tractable analytical or dynamic programming solutions to feedback policy design in fisheries, simulation-based approaches like Management Strategy Evaluation (MSE) have emerged to evaluate trade-offs among alternative feedback policies or harvest control rules \cite{punt-mse}.
In an MSE, analysts first specify policies to test a priori, and only after the specification of a policy set is simulation then used to quantify the relative performance of those policies against explicit objective(s).
While extensive application of the MSE approach has proved it to be a useful tool for informing harvest policy (\eg, \cite{edwards2016-mse}), it remains fundamentally limited to select among the a priori policy set chosen by analysts.  
Phrased differently, if a particular policy is not included a priori due to a lack of creativity on behalf of analysts or perhaps due to the constraints of national legislation (\eg, see \cite{dfo2006}), then it simply is not possible to learn whether some alternative harvest control rule might outperform those tested.  
In the context of feedback policy design this may be problematic, because feedback policy design is notoriously difficult, often counterintuitive, and because analysts typically limit themselves to 1-dimensional control rules based on stock biomass. 
	
The dynamics  of an age-structured population occurs in  high dimensional spaces, and thus in some situations it is possible for different states to correspond to the same total stock biomass.  
For example, in a standard  age-structured model, many small fish can have the same total biomass as a few large fish. Not surprisingly, managers might prescribe different management actions  in  these different contexts (see similar arguments in \cite{hilborn2002}). 
This dimensionality problem makes it difficult  to specify policies to test a priori. 
While age-structure makes it difficult to specify good policies a priori, it also implies that  managers have access to more information than total biomass in many real-world situations. 
For instance, knowing the mean fish weight in addition to total stock biomass can help the manager distinguish between a population with many small fish and one with a population biased towards large fish. 
	
Fish populations with highly variable recruitment are some of the most perplexing systems to learn from and thus in which to inform fisheries management \cite{hjort1914,caddy-gulland}, and thus might serve as a useful test case in which to explore the utility of RL for improving feedback policy design.  
In this paper we define highly variable recruitment to mean that a fish population exhibits infrequent large year classes of at least 10-50 times the long-term average recruitment level (see \cite{caddy-gulland}).  
While by definition such events are rare in any one system, a cursory review of the literature reveals that these types of fluctuations occur with some regularity in fisheries throughout the world (\eg, see \cite{fisch-etal-2019,licandeo-etal-2020}).  
For example, while better fisheries management practices helped rebuild Northeast Atlantic fish stocks, it appears that of the stocks that displayed the strongest recoveries, record large year classes occurred at low abundance that drove stock productivity upward and out of low, collapsed states (see \cite{zimmermann2019improved}).  
Similarly, Atlantic Redfish Sebastes fasciatus stocks off the eastern coast of Canada exhibit spasmodic recruitment fluctuations that make the application of standard stock assessment techniques and MSE challenging \cite{licandeo-etal-2020}.  
In inland systems, Walleye Sander vitreus in Lake Erie increased from low abundance due to a large recruitment event in 2003, and this cohort has continued to support the bulk of one of most economically valuable recreational fisheries in the world for nearly a decade \cite{schmitt-vandergoot}.  
In each of these examples, large recruitment events had marked effects on both population status and the users who rely upon those populations; however, such events are almost always written off as “environmental effects” and are notoriously difficult to predict with the reliability needed to inform feedback policy design (see \cite{punt2014, myers1998}).  
	
Little work has examined the implications of infrequent large recruitment events on the performance of feedback policies in age-structured populations, even though such policies are now considered the de facto standard for managing fisheries exploitation worldwide (\cite{silvar-viladomiu,free-etal-2022}; however see \cite{licandeo-etal-2020}).
This is noteworthy because nearly all of the theoretical work underpinning harvest control rule or feedback policy design has assumed populations exhibit uncorrelated recruitment anomalies originating from standard, stationary statistical distributions (\cite{walters1975optimal,walters-hilborn-1978,reed1979optimal}; however see \cite{parma1990experimental,hawkshaw2015harvest}). 
In this paper, we apply RL and Bayesian optimization to the problem of designing harvest control rules (HCRs) for partially-observed age-structured populations exhibiting highly variable recruitment dynamics (see Fig. 1a).  
Specifically, we use these tools to explore whether multi-dimensional control rules—particularly rules depending on the total stock biomass and the mean fish weight—can be helpful for managing age-structured, spasmodic recruiting, populations. 
We focus our case study on a recreational Walleye fishery managed via harvest lottery in Alberta, Canada (see \cite{sullivan2003}), as recent work showed these populations recovered from collapse due in part to large positive recruitment anomalies \cite{post-etal-2002,cahill2022}. 
We compare the policies obtained by numerical optimization with a rectilinear precautionary rule recommended by Canada \cite{dfo2006}.  

We evaluate HCRs with three types of utility functions: total harvest (yield maximizing), a risk-averse utility that prioritizes interannual consistency in catch, and a trophy fishing utility in which only sufficiently large fish are valued by harvesters (see Fig. 1b). 
We optimize and evaluate four kinds of HCRs: 1) constant fishing mortality (\textfmsy), 2) a rectilinear precautionary rule derived from \textfmsy and $\bmsy$ (see \cite{dfo2006}, Fig 1) for a visualization), 3) an unconstrained optimum rectilinear precautionary rule and 4) an HCR parametrized by a deep neural network using RL (see Fig. 1c).

Our findings are that:
\begin{enumerate}
\item With respect to total harvest, optimizing HCRs in classes 3 and 4 lead to considerable improvements over the simpler HCRs of classes 1 and 2. Here, in particular, the unconstrained rectilinear rule (HCR class 3) performed better than all other policies tested.
\item With respect to the risk averse utility, we observed a low performance of the constrained rectilinear policy (HCR class 2), while other HCR classes showed performances similar to each other.
\item With respect to both total harvest and the risk averse utility, our RL optimization indicates that the additional mean weight observation adds little value to the HCR, and in fact seems to hamper the RL algorithm convergence. 
\item With respect to the trophy utility function, the RL algorithm is able to advantageously use the additional mean weight observation, leading to considerably higher utility than all other HCRs we tested. 
\end{enumerate}

With respect to the trophy utility function, the RL algorithm is able to advantageously use the additional mean weight observation, leading to considerably higher utility than all other HCRs we tested. 
Through optimization we found policies that performed better than both \textfmsy and the DFO-recommended rectilinear rule. 
Second, that the \textfmsy control rule performs well with respect to the risk-averse utility—in this case, that control rule performed on par with the much more complicated HCR classes 3 and 4. 

Result 3. was rather surprising. 
Because of the complex stochasticity patterns in the model we considered, and because of the high dimensionality of our system, analytical optimality results used to inform simple HCR functional forms do not hold. 
We expected that because of this complexity, policies optimized over high-dimensional and expressive spaces would lead to improvements with respect to simple policies informed by analytical results on one-dimensional systems. 
Moreover, we expected that an additional observation on the size-structure of the population would yield valuable information for policy decisions. 
This, however, was not the case.

Result 4., in contrast to Result 3., shows that for utility functions with a strong dependence on size-structure, additional size-structure observations can allow for a better performance. 
Moreover, it demonstrates that RL can be a useful tool for optimizing HCRs when no good candidate for the functional form of the HCR is known a priori. 
The policy obtained by RL uses the combined information provided by the stock biomass and mean weight observations to more accurately time harvest pulses. Such pulse harvesting strategies have been shown to be optimal in other fishery scenarios \cite{botsford1981,darocha2013}.

%
%
%
\section{Methods}

A collection of 15-30 walleye lakes in Alberta are managed using a Special Harvest License (SHL), which enables managers to assign a Total Allowable Catch (TAC) to limit harvest, and which is unique for inland recreational fisheries in North America.  
Presently, managers use an indicator-based approach to set TACs in any particular year based on standardized gill-netting surveys which occur in fall.  
However, \cite{cahill2022} showed that harvestable surplus existed in many systems even though many management plans stated the goal of seeking to harvest  for maximum sustainable yield.  
Consequently, we seek to improve the scientific defensibility of this TAC allocation process in the Alberta SHL walleye program. 
To do so, we extended a standard age-structured population dynamics model of walleye described in detail in \cite{cahill2022} and use it to simulate realistic population dynamics.  
Briefly, we model population processes such growth in numbers at age through time as a function of Beverton-Holt stock-recruitment (see below), von Bertalanffy somatic growth in length-at-age (see \cite{cahill2020spatial}), and total mortality as an additive process assuming total instantaneous mortality Z is equal to instantaneous natural mortality M plus fishing mortality rates imposed by recreational harvesters and which is modulated by vulnerability at age.  
Unless stated otherwise parameter values for all relationships are drawn from average values estimated in \cite{cahill2020spatial}.

\subsection{Population dynamics}

A key finding of \cite{cahill2022} was that recruitment dynamics were highly variable and/or spasmodic (see also \cite{caddy-gulland}). 
Thus, we model a Walleye fishery population using a discrete-time, age-structured stochastic model with 20 age classes $(B_1, ... ,B_{20})$. 
Recruitment is modeled via the Beverton-Holt equation,
\begin{align*}
    B_1(t+1) &= r_t \alpha \frac{\ssb_t}{1 + \beta \ssb_t} - H_{t,1},\\
    B_a(t+1) &= s (B_{a-1}(t) - H_{t,a-1}), \quad \text{for } 2 \leq a < 20, \numberthis\\
    B_{20}(t+1) &= s (B_{20}(t) + B_{19}(t) - H_{t,19}).
\end{align*}
Here, $H_{t,a}$ is the harvested biomass of fish in age class $a$, $\ssb_t$ is the spawning stock biomass (\cite{cahill2022}, Eq. 8), $\alpha$ and $\beta$ are describing juvenile survival as a function of $\ssb_t$ , and $r_t$ is a random variable used to modulate recruitment from the standard Beverton-Holt model to account for highly variable and spasmodic stochastic recruitment dynamics. 
We model $r_t$ as a Bernoulli random trial with $\text{Pr}=0.025$, and when this trial is “successful” we subsequently sample rt from a uniform distribution ranging from 10 to 30. 
Phrased differently, when the Bernoulli trial for a given year is successful we multiply the Beverton-Holt recruitment prediction by a random factor of 10-30 in accordance with the types of recruitment dynamics documented in \cite{cahill2022}. 
When the Bernoulli trial is “unsuccessful,” on the other hand, we sample $r_t$ from a standard log-normal distribution like those commonly used to simulate variable recruitment (e.g., see \cite{quinn-deriso}).

The harvested biomass was calculated as
\begin{align*}
    H_{t,a} = B_a(t) F_t \vharv(a),
\end{align*}
where $F_t$ is the instantaneous  fishing mortality and $\vharv(a)$ vulnerability-at-age to harvest and was calculated as
\begin{align}
    \vharv(a) = \frac1{1 + \exp(-(a-a_{hv})/a_sl)},
\end{align}
with $a_{hv}=5$, and $a_{sl}=0.5$. 
Notice that even for a high net fishing mortality rate, e.g. $F=1$, the actual fishing mortality rate enacted on young age classes remains low, which is a common assumption in many recreational fisheries settings (e.g., see \cite{golden2022focusing}).

Simulations were run for 1000 time-steps in a heuristic attempt to capture the long-term effects of HCRs on population dynamics and performance criteria. 
Specifically, the expected number of large year classes (ie. ‘‘successful’’ Bernoulli trials for $r_t$ ) over a period of 1000 time-steps is 25, which was judged to be high enough to capture the dynamics arising from a particular HCR. 
Using the nomenclature from the RL literature, these 1000-timestep simulations are called \emph{episodes}.

\subsection{Observations}

We model observations by simulating a survey carried out by a management agency tasked with managing the fishery. 
These observations are subsequently used by the HCR to set a fishing mortality for a specific  time-step. 
We consider two types of observations. 
The first is an estimate of the stock biomass vulnerable to the management agency’s survey equipment,
\begin{align}
    \text{Stock. Biom. Obs.} = \bsurv = \sum_{a=1}^{20} B_a \vsurv(a),
\end{align}
where $\vsurv(a)$ models the vulnerability at age of the stock survey process (see \cite{cahill2022}, Eq. 15). 
A second observation we consider is the mean weight of fish in the survey,
\begin{align}
    \text{Mean Wt. Obs.} = \text{Stock. Biom. Obs.} / \nsurv,
\end{align}
where $\nsurv$ is the number of fish observed in the survey,
\begin{align}
    \nsurv = \sum_{a=1}^{20} \frac{B_a \vsurv(a)}{w_a},
\end{align}
and where $w_a$ is the average weight at age $a$.
Mean weight is easy for managers to observe and is important to consider in the context of spasmodically recruiting populations, since large recruitment events are correlated with dips in the mean weight of fish in the population.

Here we emphasize that the system is thus only partially observed \cite{memarzadeh2019}-–-while the system dynamics unfold in the high-dimensional space defined by the biomass of each age class, the manager observes only a two-dimensional summary statistics (i.e. mean biomass, total biomass via standard monitoring surveys). 
It is worth noting that while the model dynamics are Markovian in the full state space, the dynamics of these two observed states are not, making this a so-called Partially Observed Markov Decision Process, or POMDP.  
This mathematical inconvenience significantly increases the technical difficulty  of finding an optimal solution using classical tools like dynamic programming, and, historically, the analysis of ecological POMDP problems has been restricted to much smaller models  (e.g., see \cite{williams2022}).

\subsection{Utility models}

\subsection{Harvest control strategies}

\subsection{Bayesian optimization of fixed policy controls}

\subsection{Harvest control via neural network: reinforcement learning optimization}

\subsection{Policy evaluation}


%
%
%
\section{Results}


%
%
%
\section{Discussion}



%%%%%%%%%%%%%%%%%%%%%%%%%%%%%%%%%%%%%%%%%%%%%%%%%%%%%%%%%%%%%%%%%%%%%%%%%%%%%%%%%%%%%%%%%%

\bibliographystyle{apalike} 
\bibliography{fishery_refs}



\end{document}
